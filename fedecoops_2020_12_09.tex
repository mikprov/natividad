\documentclass[]{article}
\usepackage{lmodern}
\usepackage{amssymb,amsmath}
\usepackage{ifxetex,ifluatex}
\usepackage{fixltx2e} % provides \textsubscript
\ifnum 0\ifxetex 1\fi\ifluatex 1\fi=0 % if pdftex
  \usepackage[T1]{fontenc}
  \usepackage[utf8]{inputenc}
\else % if luatex or xelatex
  \ifxetex
    \usepackage{mathspec}
  \else
    \usepackage{fontspec}
  \fi
  \defaultfontfeatures{Ligatures=TeX,Scale=MatchLowercase}
\fi
% use upquote if available, for straight quotes in verbatim environments
\IfFileExists{upquote.sty}{\usepackage{upquote}}{}
% use microtype if available
\IfFileExists{microtype.sty}{%
\usepackage{microtype}
\UseMicrotypeSet[protrusion]{basicmath} % disable protrusion for tt fonts
}{}
\usepackage[margin=1in]{geometry}
\usepackage{hyperref}
\hypersetup{unicode=true,
            pdftitle={Abalone stress calculations},
            pdfauthor={Mikaela Provost},
            pdfborder={0 0 0},
            breaklinks=true}
\urlstyle{same}  % don't use monospace font for urls
\usepackage{graphicx,grffile}
\makeatletter
\def\maxwidth{\ifdim\Gin@nat@width>\linewidth\linewidth\else\Gin@nat@width\fi}
\def\maxheight{\ifdim\Gin@nat@height>\textheight\textheight\else\Gin@nat@height\fi}
\makeatother
% Scale images if necessary, so that they will not overflow the page
% margins by default, and it is still possible to overwrite the defaults
% using explicit options in \includegraphics[width, height, ...]{}
\setkeys{Gin}{width=\maxwidth,height=\maxheight,keepaspectratio}
\IfFileExists{parskip.sty}{%
\usepackage{parskip}
}{% else
\setlength{\parindent}{0pt}
\setlength{\parskip}{6pt plus 2pt minus 1pt}
}
\setlength{\emergencystretch}{3em}  % prevent overfull lines
\providecommand{\tightlist}{%
  \setlength{\itemsep}{0pt}\setlength{\parskip}{0pt}}
\setcounter{secnumdepth}{0}
% Redefines (sub)paragraphs to behave more like sections
\ifx\paragraph\undefined\else
\let\oldparagraph\paragraph
\renewcommand{\paragraph}[1]{\oldparagraph{#1}\mbox{}}
\fi
\ifx\subparagraph\undefined\else
\let\oldsubparagraph\subparagraph
\renewcommand{\subparagraph}[1]{\oldsubparagraph{#1}\mbox{}}
\fi

%%% Use protect on footnotes to avoid problems with footnotes in titles
\let\rmarkdownfootnote\footnote%
\def\footnote{\protect\rmarkdownfootnote}

%%% Change title format to be more compact
\usepackage{titling}

% Create subtitle command for use in maketitle
\providecommand{\subtitle}[1]{
  \posttitle{
    \begin{center}\large#1\end{center}
    }
}

\setlength{\droptitle}{-2em}

  \title{Abalone stress calculations}
    \pretitle{\vspace{\droptitle}\centering\huge}
  \posttitle{\par}
    \author{Mikaela Provost}
    \preauthor{\centering\large\emph}
  \postauthor{\par}
      \predate{\centering\large\emph}
  \postdate{\par}
    \date{12/09/2020}


\begin{document}
\maketitle

\hypertarget{sites}{%
\subsection{Sites}\label{sites}}

We look at four methods of calculating stress at two sites: Morro Prieto
and Punta Prieta. These sites are geographically close (\textless3 km
apart), but have very different microclimates. Punta Prieta has a lot of
temperature variability compared to Morro Prieto, but Morro Prieto is on
average warmer.

\hypertarget{method-1-short-term-variability}{%
\subsection{Method 1: short-term
variability}\label{method-1-short-term-variability}}

This method assumes that organisms experience thermal stress when
temperature they experience differs from the temperature they are
acclimatized to. Daily thermal stress increases when the difference
between experienced and expected temperature increases. Using a moving
window of 4 days, we fit a weighted linear regression using daily
temperature, and use this to estimate the `acclimatized' temperature for
each day.

In the time series plots below, color represents temperature abalone
experienced above the site-specific thermal threshold (site-specific
thermal limits are explained below). Units on the color bars is deg C.

Total stress per site is the integral of the absolute values of daily
stress (Fig 1b). We standardize each integral value by dividing it by
the number of days at each site. The units of the standardized integral
values are (deg C*day)/day --\textgreater{} deg C day.

\hypertarget{fig-1a-daily-temperatures}{%
\subsubsection{Fig 1a) Daily
temperatures}\label{fig-1a-daily-temperatures}}

\includegraphics{fedecoops_2020_12_09_files/figure-latex/unnamed-chunk-4-1.pdf}

\hypertarget{fig-1b-daily-stress-values}{%
\subsubsection{Fig 1b) Daily stress
values}\label{fig-1b-daily-stress-values}}

\includegraphics{fedecoops_2020_12_09_files/figure-latex/unnamed-chunk-5-1.pdf}

\hypertarget{method-2-degree-heating-days}{%
\subsection{Method 2: degree-heating
days}\label{method-2-degree-heating-days}}

Method 2 assumes that stress accumulates when abalone experience
temperatures above an upper thermal limit. We assume there is some local
adaptation among sites such that extreme temperatures experienced at one
site may not be considered extreme, and therefore stressful, at another
site. We define site-specific upper thermal limits as one standard
deviation above the long-term mean of temperatures observed at that
site.

Instead of using daily temperature, we use each site's climatology to
examine stress experienced above the upper thermal limit. Climatology is
calculated using the loess function in R with a 90 day smoothing window.
In the time series plots below, daily temperture (grey line),
climatology (red line), and upper thermal limit (orange dashed line) are
plotted for each site.

Total stress at each site is the integral of temperatures greater than
the site-specific thermal limit (deg C day). Just as in method 1, we
then standaridze the integrals by the number of days in time series per
site. Units of standardized integral values are deg C day.

\hypertarget{fig-2a-climatology-temperature-threshold}{%
\subsubsection{Fig 2a) Climatology \& temperature
threshold}\label{fig-2a-climatology-temperature-threshold}}

\includegraphics{fedecoops_2020_12_09_files/figure-latex/unnamed-chunk-6-1.pdf}

\hypertarget{fig-2b-daily-stress-values}{%
\subsubsection{Fig 2b) Daily stress
values}\label{fig-2b-daily-stress-values}}

\includegraphics{fedecoops_2020_12_09_files/figure-latex/unnamed-chunk-7-1.pdf}

\hypertarget{method-3-climatology-of-potential-max-temp---short-term-variability}{%
\subsection{Method 3: (climatology of potential max temp) - (short-term
variability)}\label{method-3-climatology-of-potential-max-temp---short-term-variability}}

The goal of this method is to combine stress of thermal limits and
stress associated with short-term variation in temperature. Short-term
variability can be considered `good' if temperatures are high because
the variation in temperature has a cooling off effect. But there is
still some stress associated with rapidly changing temps.

\begin{itemize}
\item
  In this menthod we calculate the potential maximum climatology.
  Climatology of the maximum values over a moving window.
\item
  Then, we integrate the maximum climatology values above the
  site-specific thermal threshold. And standardize this integral by the
  number of days in temperature time series.
\item
  Short-term variability is considered `good' in this case and therefore
  reducing stress, so we subtract the amount of short-term variability
  stress as calculated in method 1 from the standardized integrals.
\end{itemize}

\hypertarget{fig-3a-max-climatology---short-term-variability-stress}{%
\subsubsection{Fig 3a) (max climatology) - (short-term variability
stress)}\label{fig-3a-max-climatology---short-term-variability-stress}}

Variability is a good thing when temperature is high. Abalone are
`cooled off'.

\includegraphics{fedecoops_2020_12_09_files/figure-latex/unnamed-chunk-8-1.pdf}

\hypertarget{fig-3b-daily-stress-values}{%
\subsubsection{Fig 3b) Daily stress
values}\label{fig-3b-daily-stress-values}}

Stress = (deg C of max climatology above threshold) - (stress due to
short-term variability)

\includegraphics{fedecoops_2020_12_09_files/figure-latex/unnamed-chunk-9-1.pdf}

\hypertarget{comparing-3-stress-values}{%
\subsubsection{Comparing 3 stress
values}\label{comparing-3-stress-values}}

\begin{itemize}
\item
  Method 1: Short-term variability
\item
  Method 2: Cumulative temp of mean climatology above thermal threshold
\item
  Method 3: (max climatology) - (short-term variability)
\end{itemize}

\includegraphics{fedecoops_2020_12_09_files/figure-latex/unnamed-chunk-10-1.pdf}

\hypertarget{method-4-thermal-performance-curve}{%
\subsection{Method 4: Thermal performance
curve}\label{method-4-thermal-performance-curve}}

We assume abalone operate under a specific thermal performance curve
with a thermal limit (Tmax). The thermal response curve is adapted from
{[}Barneche et al 2014{]}
(\url{https://onlinelibrary-wiley-com.stanford.idm.oclc.org/doi/full/10.1111/ele.12309?casa_token=eU9n50Xo5QwAAAAA\%3APi-WSH_dZ1AtzhuWL0GzM-VUmTep3yQrmLtkVX6hknA4AWliVIK0HtL29MACgNFMkJtdCmH5K9_L0gj_}).
The optimal temperature is set to the mean of the data from a site. We
then normalize and invert the thermal performance curve so that the
stress factor is zero at the optimal temperature and one above Tmax.
Then to calculate stress, the difference between the optimal temperature
and the temperature at each time is multiplied by the stress factor. We
then take the absolute value and integrate to get an overall stress in
deg C days.


\end{document}
